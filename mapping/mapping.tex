\chapter{Mapping}

\section{Introduction}
The path planning and localization methods we have seen so far require knowledge
of the robot's environment.  Some of these methods rely on a map or grid.  
How does this map get created and
what data structures are used to represent this data?  The success of localization and planning algorithms is
highly dependent on the quality and features that a map provides.  Consider the case of the 
developing a self-driving car or vehicle, where an accurate map that is dynamic in nature (other
obstacles are moving) is critical for success.

So how do these maps get created?  Can a robot create these maps on its own?  This chapter 
introduces \textit{robotic mapping} algorithms where an autonomous robot builds a map for use in other tasks like 
planning and localization.  

\section{Map Representations}

Two popular data structures for representing maps are \textit{metric maps} and \textit{topological maps}.  In this
section, we will focus on metric maps.

\subsection{Metric Maps}
Metric maps place place onto 2D or 3D maps using coordinates.  While other techniques exist, this
most closely mimics other maps that we use (such as a map for driving or a floor plan of a house).  
These maps get their names from the fact that precise locations allow for distance and other 
``metric" computations.  

\subsubsection{Occupancy Grids}
Occupancy grids place either a 2d or 3d grid over a space.  This discretization marks each grid cell as occupied or free.  
The application and robot environment would dictate the granularity of the grid structure.  


\subsubsection{Quad-Trees and Oct-Trees}
Occupancy grids work well but have some inefficiencies.  Consider an environment where the
there are few obstacles and landmarks.  Large collections of grid cells would be empty.  This problem
is amplified when considering a grid of fine granularity.  

A more efficient approach can be seen in the \textit{quad} and \textit{oct-tree} data structures, which
were made popular in the graphics world.  Consider the case of the quad-tree, which is utilized 
in 2D maps.  Define the dimension of the smallest grid cell and call it $\delta$. The entire space is divided into 4 quadrants. 
Consider each of the four quadrants: if the quadrant is free of obstacles, no further work is necessary for that quadrant.  If it
is not obstacle free, that quadrant is divided into 4 quadrants and the method repeats until either the quadrant is obstacle free
or the desired minimum granularity $\delta$ is reached.  Oct-trees represent the same idea except in 3D space.

This approach is more efficient since it allows large sections of unoccupied space to be clustered together.

%\subsection{Topological Maps}


%\section{Mapping with Perfect Odometry}

\section{SLAM}
If a robot is going to construct a map, part of this process certainly would need to keep track of
where it is on the map it is constructing. But to keep track of where the robot is we need a map?
This seems like the chicken and the egg problem.  However, tackling both mapping and localization
at the same time seems natural and crucial in many practical robotic applications. 

The simultaneous localization and mapping algorithm known as \textit{SLAM} addresses this challenge.  
One of the first papers introducing this idea was Durrant-Whyte paper in 1996.



Cite: H. Durrant-Whyte, D. Rye and E. Nebot, "Localization of automatic guided vehicles", Robotics Research: The 7th International Symposium (ISRR'95), pp. 613-625, 1996.

\textit{SLAM} makes an estimate of both the robot's position ($x$) and the location of all known landmarks ($m$).  As the robot
explores the environment, a history of all control inputs ($u$) and landmark observations ($z$) are kept.  All of
these variables can be indexed by $k$, which represents the time index of the observations.  \textit{SLAM}
represents the belief of the location of the robot and the landmarks at time $k$ as a probability
distribution:


\begin{center}
\begin{equation}
P(x_k,m | Z_{0:k}, U_{0:k}, x_0)
\end{equation}
\end{center}

Things to add:
\begin{itemize}
\item time-update
\item measurement-update
\item extended Kalman filter slam (EKF-SLAM)
\item Rao-Blackwellized particle filters
\item challenges with loop closure

\end{itemize}

%\subsection{EKF-SLAM}

%\subsection{Rao-Blackwellized Particle Filters}
